\subsection{Les groupes}
Chaque utilisateur Watchdog enregistr� � la possibilit� d'appartenir � des groupes. A chaque groupe doit �tre associ�e une entit�e physique, technique, ou morale. Par exemple, dans le cas d'une entreprise, il est conseill� de poss�der les groupes {\it Direction}, {\it Annuaire}, et possible d'avoir un groupe {\it Entr�e 3} pour limiter les acc�s � l'entr�e 3 du site.

\begin{description}
\item[Groupe] Un {\bf groupe} permet de regrouper dans une m�me entit� un ensemble d'utilisateurs ayant des droits identiques.
\end{description}



Ces groupes ont les particularit�s de ne pouvoir subir aucune modification quelle qu'elle soit, et de ne pas pouvoir �tre �limin�s.

\subsubsection{Les groupes normaux}
Seuls une cat�gorie d'utilisateurs ont la possibilit�s de g�rer la database {\bf groups} associ�es aux groupes Watchdog. Ces utilisateurs sont tous dans le groupe syst�me {\bf Admin-userDB}. Leurs diff�rents actions possibles sont pr�sent�es dans la table \ref{Action_groupe}

\begin{tablseb}{Actions sur les groupes}{ll}{Actions & D�signations}{Action_groupe}
Ajouter & Cr�ation d'un nouveau groupe \\
Enlever & Elimination d'un groupe existant \\
Editer  & Modification d'un groupe existant \\
\end{tablseb}

