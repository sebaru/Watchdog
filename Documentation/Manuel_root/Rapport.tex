\documentclass[10pt,a4paper]{report}
\usepackage[francais]{babel}
\usepackage[T1]{fontenc}
\usepackage{graphicx}
\pagestyle{headings}
\makeindex

\newcommand{\dessin}[2]{\begin{figure}[h] \begin{center} 
\includegraphics{#1.eps} \end{center} \caption{#2}
\label{#1} \end{figure}}

\newenvironment{tabuseb}[2]
{ \begin{center}
  \begin{sf}
  \begin{tabular}[htbp]{#1}
  #2 \\
  \hline
}
{
  \end{tabular}
  \end{sf}
  \end{center}
}

\newenvironment{tablseb}[4]
{ \begin{table}[htbp]
  \caption{#1}
  \label{#4}
  \begin{tabuseb}{#2}{#3}
}
{
  \end{tabuseb}
  \end{table}
}

\begin{document}
\title{Watchdog V2.0\\Support Technique}
\author{S�bastien Lefevre}
%\date{26 novembre 2002}
\maketitle
\tableofcontents
\listoftables

\chapter{Introduction/Pr�sentation}
Watchdog est un syst�me de controle/commande d'un habitat, extensible aux petites et moyennes entreprises. Articul� autour d'un mod�le client-serveur, un protocole de communication est n�cessaire pour acheminer les informations au serveur, et commander les diff�rents actionneurs du syst�me. Ce support se d�compose en plusieurs sections d�crivant les diff�rentes possibilit�s de configuration que poss�de l'utilisateur privil�gi� {\bf root}, ensuite qu'un chapitre entier consacr� au protocole Watchdog lui-m�me et enfin un chapitre d�di� � l'installation d'un syst�me {\bf Watchdog}.



%\input{Administration}
%\chapter{Protocole Watchdog}
Le protocole Watchdog en lui-m�me est simple de part l'utilisation de paquet relativement simples. Pour la suite, nous considerons qu'un paquet watchdog est r�duit aux champs pr�sent�s dans la table \ref{Champ_paquet}.

\begin{tablseb}{Champs d'un paquet Watchdog}{ll}{Noms & D�signations}{Champ_paquet}
tag & le type du paquet \\
type\_donn�es & le type des donn�es transmises \\
donn�es & les donn�es elles-m�mes \\
\end{tablseb}


%\input{Installation}
%*****************************************************
% Conclusion
%*****************************************************
\chapter{Conclusions}
Ce projet nous a permis de r�fl�chir sur un probl�me int�ressant croisant le domaine de la compression et celui des syst�mes embarqu�s.
A partir de simples �quations, nous avons r�ussi mettre en oeuvre trois syst�mes, selon trois
niveaux de conception, fonctionnels et efficaces. Gr�ce au cours, nous sommes parvenus � simplifier
de mani�re spectaculaire la premi�re version de notre {\bf section}, passant de plus de 14 registres
� seulement 3 bancs de registres.

Plusieurs probl�mes ont �t� d�couverts puis r�solus lors de la r�alisation de ce projet.
\begin{itemize}
\item les probl�mes de co�t $\rightarrow$ s�paration des automates de commande de chaque niveau de conception.
\item les probl�mes de s�quentialisation $\rightarrow$ r�solus gr�ce � l'ajout des signaux START et STOP dans
      chacun des niveaux de conception.
\item les probl�mes d'optimisations $\rightarrow$ r�solus gr�ce au cours sur les Syst�mes Embarqu�s.
\item les probl�mes de programmation $\rightarrow$ r�solus en r�fl�chissant longtemps\ldots \\
\end{itemize}

Avec le d�tail des architectures pr�sent�es dans ce rapport, le passage de notre syst�me en VHDL devrait �tre facile
pour toute personne en ayant le besoin.\\
Nous vous remercions pour l'attention port� � ce projet, nous restons � votre disposition pour toute
demande de renseignements compl�mentaires :\\

\begin{center}
Collin Moretto ( {\it netboulot@meloo.com} ) \\
S�bastien LEFEVRE ( {\it lefevre.s22@wanadoo.fr} )\\
\end{center}

\end{document}
