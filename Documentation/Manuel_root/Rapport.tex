\documentclass[10pt,a4paper]{report}
\usepackage[francais]{babel}
\usepackage[T1]{fontenc}
\usepackage{graphicx}
\pagestyle{headings}
\makeindex

\newcommand{\dessin}[2]{\begin{figure}[h] \begin{center} 
\includegraphics{#1.eps} \end{center} \caption{#2}
\label{#1} \end{figure}}

\newenvironment{tabuseb}[2]
{ \begin{center}
  \begin{sf}
  \begin{tabular}[htbp]{#1}
  #2 \\
  \hline
}
{
  \end{tabular}
  \end{sf}
  \end{center}
}

\newenvironment{tablseb}[4]
{ \begin{table}[htbp]
  \caption{#1}
  \label{#4}
  \begin{tabuseb}{#2}{#3}
}
{
  \end{tabuseb}
  \end{table}
}

\begin{document}
\title{Watchdog V2.0\\Support Technique}
\author{S�bastien Lefevre}
%\date{26 novembre 2002}
\maketitle
\tableofcontents
\listoftables

\chapter{Introduction/Pr�sentation}
Watchdog est un syst�me de controle/commande d'un habitat, extensible aux petites et moyennes entreprises. Articul� autour d'un mod�le client-serveur, un protocole de communication est n�cessaire pour acheminer les informations au serveur, et commander les diff�rents actionneurs du syst�me. Ce support se d�compose en plusieurs sections d�crivant les diff�rentes possibilit�s de configuration que poss�de l'utilisateur privil�gi� {\bf root}, ensuite qu'un chapitre entier consacr� au protocole Watchdog lui-m�me et enfin un chapitre d�di� � l'installation d'un syst�me {\bf Watchdog}.

\section{Architecture g�n�rale}
Nous pouvons scind� un syst�me {\bf Watchdog} selon 3 entit�s distinctes sur le papier, mais qui peuvent �tre identiques physiquement. Ces 3 entit�s sont: le serveur de base de donn�es, le serveur Watchdog, et les clients. Nous avons donc l'architecture de la figure \ref{Schema_principe}. 

\dessin{Schema_principe}{L'architecture Watchdog}

Remarque: Il se peut que le serveur Watchdog et celui de base de donn�es soient effectivement sur la m�me machine physique.

%******************* Conventions *************************
\section{Conventions}
Nous aurons besoin tout au long de ce rapport de quelques conventions que nous pr�cisons dans la table \ref{conventions}.

\begin{tablseb}{Conventions du rapport}{l}{}{conventions}
 - Une entit� sera repr�sent� graphiquement dans un carr� aux coins arrondis.
\end{tablseb}

%**********************************************************
\section{Les databases Watchdog}
Le syst�me Watchdog doit pouvoir communiquer avec une base de donn�es appel�e {\bf WatchdogDB} par d�faut, 
dans laquelle les tables suivantes doivent �tre pr�sentes (table \ref{DBs}).

\begin{tablseb}{Tables de la base WatchdogDB}{ll}{Noms & D�signations}{DBs}
users  & Les utilisateurs du syst�me \\
groups & Les groupes du syst�me \\
msgs   & Les messages au fil de l'eau du syst�me \\
icons  & La base graphique du syst�me \\
syns   & L'ensemble des synoptiques du syst�me \\
\end{tablseb}
Nous avons donc le schema de principe de la figure \ref{Schema_db}.
\dessin{Schema_db}{Schema de principe de la DB Watchdog}

%**********************************************************
\section{Utilisateurs, groupes, droits et privil�ges}
Chaque utilisateur du syst�me Watchdog doit �tre d�clar� dans la base {\bf users} et poss�de un mot de passe.

Chaque utilisateur Watchdog enregistr� � la possibilit� d'appartenir � des groupes. A chaque groupe doit �tre associ�e une entit�e physique, technique, ou morale. Par exemple, dans le cas d'une entreprise, il est conseill� de poss�der les groupes {\it Direction}, {\it Annuaire}, et possible d'avoir un groupe {\it Entr�e 3} pour limiter les acc�s � l'entr�e 3 du site.

\begin{description}
\item[Groupe] Un {\bf groupe} permet de regrouper dans une m�me entit� un ensemble d'utilisateurs ayant des droits identiques.
\item[Utilisateur] Un {\bf utilisateur} est un triplet (nom, mot de passe, groupes). 
\end{description}

Un exemple d'utilisateurs et de groupes est pr�sent� dans la figure \ref{Exemple_utilgroupe}.
\dessin{Exemple_utilgroupe}{Exemple d'utilisateurs et de groupes}

Dans cet exemple nous voyons que l'utilisateur Bruno fait parti de la direction et ne peut rentrer que par l'entr�e 1. Jean est aussi de la direction mais rentre par l'entr�e 2. Olivier fait parti du service finance et peut entrer par l'entr�e 1 aussi bien que par l'entr�e 2.

\subsection{Les utilisateurs syst�mes}
Certains utilisateurs (ou comptes) sont dits {\bf syst�mes}. Ce sont des utilisateurs qu'il n'est pas possible de modifier ou d'�liminer. Par exemple, le compte {\bf root} est un compte syst�me et ne peut pas �tre d�truit. Ces utilisateurs ont des responsabilit�s exigeant des droits d'administrateurs et c'est pourquoi ils doivent �tre utilis�s avec parcimonie.
Les utilisateurs syst�mes sont pr�sent�s dans la table \ref{Util_systeme}
 
\begin{tablseb}{Les utilisateurs syst�mes}{lp{250pt}}{Noms & D�signations}{Util_systeme}
root     &  Administration g�n�rale syst�me \\
\end{tablseb}

\subsection{Les groupes syst�mes}
Certains groupes sont dits {\bf syst�mes}. Ces groupes sont pr�sent�s dans la table \ref{Groupe_systeme}

\begin{tablseb}{Les groupes syst�mes}{lp{250pt}}{Noms & D�signations}{Groupe_systeme}
Everybody     &  Tout le monde appartient � ce groupe \\
Admin-userDB  &  Ses utilisateurs ont l'autorisation d'acc�s aux DBs {\bf users} et {\bf groups} \\
Admin-msgDB   &  Ses utilisateurs ont l'autorisation d'acc�s � la DB {\bf msgs} \\
Admin-iconDB  &  Ses utilisateurs ont l'autorisation d'acc�s � la DB {\bf icons} \\
Admin-synopDB &  Ses utilisateurs ont l'autorisation d'acc�s � la DB {\bf synops} \\
Log           &  Ses utilisateurs ont l'autorisation d'acc�s � la DB {\bf logs} \\
\end{tablseb}

Comme indiqu�, � chaque groupe syst�me du type Admin, est associ�e une table de la database WatchdogDB. Cela permet un controle simple et exhaustif des acc�s en {\bf �criture} � ses diff�rentes tables.

\subsubsection{Le groupe Everybody}
Ce groupe est sp�cial car il est le groupe dans lequel nous pouvons trouver tous les utilisateurs du syst�me, sans exception. Bien s�r, l'�limination du groupe Everybody d'un utilisateur est impossible.

\subsubsection{Le groupe Log}
Seule la table {\bf logs} est une table � {\bf lecture} control�e. En effet, tous les utilisateurs n'ont pas forcement le droit de lire les logs du syst�mes. Afin de les autoriser � lire les logs, il faudra donc les placer dans le groupe {\bf Log}.


%\input{Administration}
%\chapter{Protocole Watchdog}
Le protocole Watchdog en lui-m�me est simple de part l'utilisation de paquet relativement simples. Pour la suite, nous considerons qu'un paquet watchdog est r�duit aux champs pr�sent�s dans la table \ref{Champ_paquet}.

\begin{tablseb}{Champs d'un paquet Watchdog}{ll}{Noms & D�signations}{Champ_paquet}
tag & le type du paquet \\
type\_donn�es & le type des donn�es transmises \\
donn�es & les donn�es elles-m�mes \\
\end{tablseb}


%\input{Installation}
%*****************************************************
% Conclusion
%*****************************************************
\chapter{Conclusions}
Ce projet nous a permis de r�fl�chir sur un probl�me int�ressant croisant le domaine de la compression et celui des syst�mes embarqu�s.
A partir de simples �quations, nous avons r�ussi mettre en oeuvre trois syst�mes, selon trois
niveaux de conception, fonctionnels et efficaces. Gr�ce au cours, nous sommes parvenus � simplifier
de mani�re spectaculaire la premi�re version de notre {\bf section}, passant de plus de 14 registres
� seulement 3 bancs de registres.

Plusieurs probl�mes ont �t� d�couverts puis r�solus lors de la r�alisation de ce projet.
\begin{itemize}
\item les probl�mes de co�t $\rightarrow$ s�paration des automates de commande de chaque niveau de conception.
\item les probl�mes de s�quentialisation $\rightarrow$ r�solus gr�ce � l'ajout des signaux START et STOP dans
      chacun des niveaux de conception.
\item les probl�mes d'optimisations $\rightarrow$ r�solus gr�ce au cours sur les Syst�mes Embarqu�s.
\item les probl�mes de programmation $\rightarrow$ r�solus en r�fl�chissant longtemps\ldots \\
\end{itemize}

Avec le d�tail des architectures pr�sent�es dans ce rapport, le passage de notre syst�me en VHDL devrait �tre facile
pour toute personne en ayant le besoin.\\
Nous vous remercions pour l'attention port� � ce projet, nous restons � votre disposition pour toute
demande de renseignements compl�mentaires :\\

\begin{center}
Collin Moretto ( {\it netboulot@meloo.com} ) \\
S�bastien LEFEVRE ( {\it lefevre.s22@wanadoo.fr} )\\
\end{center}

\end{document}
